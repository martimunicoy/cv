%-------------------------------------------------------------------------------
%	DEFINITIONS
%-------------------------------------------------------------------------------

\def\angle{0}
\def\radius{1.5}
%\def\cyclelist{{"piechart1", "piechart2", "piechart3", "piechart4", "piechart5", "piechart6", "piechart7", "piechart8", "piechart9", "piechart10"}}
%\def\cyclelist{{"piechartred1", "piechartred2", "piechartred3", "piechartred4", "piechartred5", "piechartred6", "piechartred7", "piechartred8", "piechartred9", "piechartred10"}}
\def\cyclelist{{"piechartorange1", "piechartorange2", "piechartorange3", "piechartorange4", "piechartorange5", "piechartorange6", "piechartorange7", "piechartorange8", "piechartorange9", "piechartorange10"}}
% \def\cyclelist{{"piechartmyblues1", "piechartmyblues2", "piechartmyblues3", "piechartmyblues4", "piechartmyblues5", "piechartmyblues6", "piechartmyblues7", "piechartmyblues8", "piechartmyblues9", "piechartmyblues10"}}

%-------------------------------------------------------------------------------
%	CONTENT
%-------------------------------------------------------------------------------
\begin{minipage}[t]{0.5\textwidth}
\cvsection{Programming skills}
\begin{cvskills}

\vspace{1em}

\newcount\cyclecount \cyclecount=-1
\newcount\ind \ind=-1
\begin{tikzpicture}[nodes = {}]
  \foreach \percent/\name in {
      35/Python,
      30/C and C++,
      15/GitHub,
      5/LaTeX,
      5/GNU Bash,
      5/SQL,
      5/HTML5 and CSS
    } {
      \ifx\percent\empty\else               % If \percent is empty, do nothing
        \global\advance\cyclecount by 1     % Advance cyclecount
        \global\advance\ind by 1            % Advance list index
        \ifnum10<\cyclecount                 % If cyclecount is larger than list
          \global\cyclecount=0              %   reset cyclecount and
          \global\ind=0                     %   reset list index
        \fi
        \pgfmathparse{\cyclelist[\the\ind]} % Get color from cycle list
        \edef\color{\pgfmathresult}         %   and store as \color
        % Draw angle and set labels
        \draw[fill={\color},draw={\color}] (0,0) -- (\angle:\radius)
          arc (\angle:\angle+\percent*3.6:\radius) -- cycle;
        \node[pin=\angle+0.5*\percent*3.6:\small \name]
          at (\angle+0.5*\percent*3.6:\radius) {};
        \pgfmathparse{\angle+\percent*3.6}  % Advance angle
        \xdef\angle{\pgfmathresult}         %   and store in \angle
      \fi
    };
\end{tikzpicture}

%---------------------------------------------------------
\end{cvskills}

\end{minipage}
\begin{minipage}[t]{0.5\textwidth}


\cvsection{Scientific software}
\begin{cvprograms}

\vspace{1em}

\newcount\cyclecount \cyclecount=-1
\newcount\ind \ind=-1
\begin{tikzpicture}[nodes = {}]
  \foreach \percent/\name in {
      30/Maestro,
      15/UCSF Chimera,
      15/Glide,
      10/Desmond,
      10/VMD,
      10/OpenMM,
      5/MDTraj,
      5/GOLD,
      5/AmberTools
    } {
      \ifx\percent\empty\else               % If \percent is empty, do nothing
        \global\advance\cyclecount by 1     % Advance cyclecount
        \global\advance\ind by 1            % Advance list index
        \ifnum10<\cyclecount                 % If cyclecount is larger than list
          \global\cyclecount=0              %   reset cyclecount and
          \global\ind=0                     %   reset list index
        \fi
        \pgfmathparse{\cyclelist[\the\ind]} % Get color from cycle list
        \edef\color{\pgfmathresult}         %   and store as \color
        % Draw angle and set labels
        \draw[fill={\color},draw={\color}] (0,0) -- (\angle:\radius)
          arc (\angle:\angle+\percent*3.6:\radius) -- cycle;
        \node[pin=\angle+0.5*\percent*3.6:\small \name]
          at (\angle+0.5*\percent*3.6:\radius) {};
        \pgfmathparse{\angle+\percent*3.6}  % Advance angle
        \xdef\angle{\pgfmathresult}         %   and store in \angle
      \fi
    };
\end{tikzpicture}

\end{cvprograms}

\end{minipage}